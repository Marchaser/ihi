\documentclass[endfloat]{ecca}
\usepackage[T1]{fontenc}
\usepackage[latin1]{inputenc}
\usepackage{csquotes}
\MakeInnerQuote{"}

\usepackage{hyperref}
\hypersetup{%
   colorlinks = {true},
   urlcolor = {blue},
   linkcolor = {black},
   citecolor = {black},
   pdfauthor = {Arne Henningsen},
   pdftitle = {Testing LaTeX class and BibTeX style for the
      Journal Economica (ECCA)},
   pdfkeywords = {Economica, BibTeX, LaTeX}
}

\usepackage{multido}

\title{Testing \LaTeX{} class and Bib\TeX{} style for the
   Journal ``Economica'' (ECCA)}
\author{Arne Henningsen \and{} Jim Nobody}
\keywords{Economica, BibTeX, LaTeX}
\jelclass{A1, B2, C3}

\begin{document}

\maketitle

\begin{abstract}
\multido{}{15}{This is an abstract. }
\end{abstract}

\section*{Introduction}

The first section header should not be numbered.
Hence, use the starred \texttt{\textbackslash{}section} command
for the header of this section,
i.e.\ \texttt{\textbackslash{}section$^*$\{Introduction\}}.
Type all (foot)notes at the end of the paper.%
\footnote{
\multido{}{5}{This is automatically done with the "endnote" package. }
} 
Place (long) tables and figures at the end of the paper
(see figure~\ref{fig:dummy} and table~\ref{tab:citations}).
You can use the class option "\texttt{endfloat}" to move
all float environments (figures and tables)
to the end of the output file,
i.e. use \texttt{\textbackslash{}documentclass[endfloat]\{ecca\}}.

\begin{figure}[htbp]
\centering
\fbox{\parbox{0.6 \textwidth}{\centering
   \vspace{0.2 \textwidth}
   This is not a figure.
   \vspace{0.2 \textwidth}
}}
\medskip\\
Note: Do not forget to center your figures,
i.e.\ use command \texttt{\textbackslash{}centering}.
\caption{Dummy figure}
\label{fig:dummy}
\end{figure}

\begin{figure}[htbp]
\centering
\fbox{\parbox{0.6 \textwidth}{\centering
   \vspace{0.2 \textwidth}
   This is not a figure, too.
   \vspace{0.2 \textwidth}
}}
\caption{Figure with \multido{}{40}{very } long title}
\label{fig:long-title}
\end{figure}


\section{Manuscript Formatting}
The manuscript formatting instructions are available at
\url{http://www.blackwellpublishing.com/ecca}.


\section{Citations}
\subsection{Citations in Text}

\citeauthor{brown00} in a paper on \ldots

\citet[p.~12]{brown00} has shown that \ldots

A proof is given by \citet{jones98}.

An overview is available in table~\ref{tab:citations}.

\begin{table}[htbp]
\centering
\caption{Citations}
\label{tab:citations}
\begin{tabular}{lll}
\hline
Author(s) & Year & Citation\\
\hline
\citeauthor{brown00} & \citeyear{brown00} & \citet{brown00}\\
\citeauthor{jones99} & \citeyear{jones99} & \citet{jones99}\\
\citeauthor{jones99a} & \citeyear{jones99a} & \citet{jones99a}\\
\citeauthor{brown00a} & \citeyear{brown00a} & \citet{brown00a}\\
\citeauthor{jones98} & \citeyear{jones98} & \citet{jones98}\\
\citeauthor{allen95} & \citeyear{allen95} & \citet{allen95}\\
\citeauthor{atkinson92} & \citeyear{atkinson92} & \citet{atkinson92}\\
\citeauthor{bernanke88} & \citeyear{bernanke88} & \citet{bernanke88}\\
\citeauthor{bernanke95} & \citeyear{bernanke95} & \citet{bernanke95}\\
\citeauthor{bernanke96} & \citeyear{bernanke96} & \citet{bernanke96}\\
\hline
\end{tabular}
\medskip \\
Notes: Do not use vertical lines in tables;
do not forget to center your tables,
i.e.\ use command \texttt{\textbackslash{}centering}.
\end{table}


\subsection{Citations in Parenthesis}

This method has been criticised \citep{jones99a, jones99, brown00a}.


\section{Equations}
All displayed equations should be left-justified
and --- where necessary --- numbered consecutively (on the left).
\begin{equation}
y = a + X b
\end{equation}
where $a$ is a scalar,
$y$ and $b$ are vectors,
and $X$ is a matrix.

Now, we have an unnumbered equation
\begin{equation*}
\theta = \alpha + \Psi \beta
\end{equation*}
where $\alpha$ is a scalar,
$\theta$ and $\beta$ are vectors,
and $\Psi$ is a matrix.


\section{Quotation Marks}
The "Economica" requires single quotes.
You can get pretty (typographically correct) quotation marks
by adding the commands
\verb!\usepackage{csquotes}!
and
\verb!\MakeInnerQuote{"}!
in your preamble and use the \verb!"! sign for quation marks in your text.


\section*{Acknowledgements}

The Bib\TeX{} code to replace repeated author names by multiple dashes
in the list of references has been kindly provided by
Joseph A.\ Wright (\url{joseph.wright@morningstar2.co.uk}).


\listofendnotes

\appendix

\section{Websites of the "Economica"}
\begin{itemize}
\item \url{http://www.blackwellpublishing.com/ecca}
\item \url{http://darp.lse.ac.uk/Frankweb/Economica/EconomicaHome.htm}
\end{itemize}

\section{Proofs}
\begin{equation*}
14 = 2 \cdot 7  = 2 ( 3 + 4 ) = 2 \cdot 3 + 2 \cdot 4 = 6 + 8 = 14
\end{equation*}


\nocite{*}

\bibliographystyle{ecca}
\bibliography{ecca-ex}

\end{document}
